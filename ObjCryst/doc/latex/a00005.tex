\-If you have a question you can {\tt drop me an email}\subsection{\-History}\label{a00005_history}
\begin{DoxyParagraph}{1.1 }
\begin{DoxyItemize}
\item \-I\-N\-C\-O\-M\-P\-A\-T\-I\-B\-L\-E \-C\-H\-A\-N\-G\-E\-S\-:\-Global\-Optim\-Obj has been replaced by \-Monte\-Carlo\-Obj, \-Refinable\-Obj\-::\-Get\-Class\-Name now returns a \char`\"{}const string\&\char`\"{}, and \-Refinable\-Obj\-::\-Begin\-Optimization has two parameters, to enable approximations and restraints \item now one can choose either float or double for precision. \-The default is float for global optimization. \item \-Changed \-Refinable\-Obj\-::\-Begin\-Optimization to pass two flags to enable approximations (faster) and restraints before beginning an optimization. \item \-Z\-Scatterer\-: worked on \-Z\-Scatterer\-::\-Global\-Optim\-Random\-Move(), to give the possibility of making 'smart' moves, especially for molecules. \item \-Scatterer\-: removed the \-Z\-Scatterer\-::\-Update() function from the base \-Scatterer class, as previously scheduled. \item \-Optimization\-Obj\-: \-Forked the algorithms classes, with a base \-Optimization\-Obj class, derived (currently) to a \-Monte\-Carlo\-Obj class (which replaces the old \-Global\-Optimi\-Obj class, for simulated annealing and parallel tempering). \-Also added a tentative \-Genetic\-Algorithm class, still in very early development and not usable yet for the common mortal. \item restraints\-: added base \-Restraint class, not really used so far. \item cleaned up a bit wx\-Cryst classes in the hope to remove any \-G\-U\-I bugs. \-Still some crashes under windows, unfortunately...\end{DoxyItemize}

\end{DoxyParagraph}
\begin{DoxyParagraph}{1.02(2001-\/nov 12th)}
\begin{DoxyItemize}
\item \-Powder\-Pattern\-: \-Added the input format for \-Sietronics (.cpi) files. \-Now importing data will allow null points without crashing. \-Also when no phase (background, crystal) has been added to the pattern, the calculated pattern returns a constant value (1.). \item \-Z\-Scatterer\-: corrected a bug in the fractionnal coordinates calculation for mono/triclinic unit cells (thanks \-Mark \-Edgar).\end{DoxyItemize}

\end{DoxyParagraph}
\begin{DoxyParagraph}{0.9.1(2001-\/09-\/20-\/)}
\begin{DoxyItemize}
\item \-Z\-Scatterer\-: corrected a nasty bug(thanks Yuri Andreev), which made the \-Z-\/\-Matrix interpretation completely wrong. \item wx\-Ref\-Par\-: now all refinable parameters have a local menu which can be used to remove or change limits. \item wx\-Z\-Scatterer\-: now can globally change relative limits to bond lengths and angles.\end{DoxyItemize}

\end{DoxyParagraph}
\begin{DoxyParagraph}{0.9(2001-\/09-\/18-\/first \-F\-O\-X \-Release)}
\begin{DoxyItemize}
\item wx\-Cryst is now working, allowing the first compilation of \-Fox. \item \-The data format to save files has been changed to an xml format (see {\tt http\-://www.\-w3.\-org/\-X\-M\-L/}). \item the definition of the refinable parameter types (\-Ref\-Par\-Type) are now made using pointers, in a tree-\/like fashion. \-This allows an easier modification of status for groups of parameters. \item \-Lots of other modifications...\end{DoxyItemize}

\end{DoxyParagraph}
\begin{DoxyParagraph}{0.5(july 2001)}
\begin{DoxyItemize}
\item \-Design largely improved for reusability (based on inheritance). \item wx\-Cryst (the \-G\-U\-I-\/\-Graphical \-User \-Interface) part is in fast-\/moving development, mostly working, but not included here as code is still ugly and its design is not stable. \-As of 10/july/2001, \-I can launch the program, create a crystal, add a \-Powder\-Pattern object, a \-Global\-Optimization object and do the global optimization while looking at the 'live' evolution of the \-Crystal \-Structure (3\-D display with \-Open\-G\-L) and of the \-Powder\-Pattern ! \item \-Saving structures, data to files has been implemented. \-See \-Refinable\-Obj\-::\-X\-M\-L\-Output(). \-Refinement flags are not saved (yet). \item \-Changed all float to \-R\-E\-A\-L precision. \-Performance hit=+15\% on structure factor, and +25\% on full powder spectrums.\end{DoxyItemize}

\end{DoxyParagraph}
\begin{DoxyParagraph}{0.6(march 2001)}
\-First public release.
\end{DoxyParagraph}
\begin{DoxyParagraph}{0.5 (february, 2001)\-: }
\begin{DoxyItemize}
\item \-Z\-Polyhedron \-Added a \-Z\-Polyhedron class, based on the \-Zscatterer objects, to describe (almost) regular polyhedra. \item \-Interatomic distance calculations \-Improved speed computation of the internel distance table, using non-\/symetrical atoms and the \-Asymmetric unit cell. \item \-Asymmetric\-Unit \-Cell \-Finally implemented the \-Asymmetric\-Unit \-Cell class, to improve distance calculation. \-Currently, upon \-Spacegroup construction, the (or one of the) smallest parallelepiped (with sides parallel to the crystallographic axes) including an \-Asymmetric unit is computed, using steps of 1/24. (\-So technically it is not an asymmetric unit cell, if no asymmetric unit cell exists for the group with all vertices parallel to the unit cell faces)\end{DoxyItemize}

\end{DoxyParagraph}
\begin{DoxyParagraph}{0.4 (january 2001)\-:}
\begin{DoxyItemize}
\item \-Z\-Scatterer \-Added a \-Zscatterer class, were cluster of atoms or molecules can be defined using a \-Z-\/\-Matrix description (see {\tt http\-://chemistry.\-umeche.\-maine.\-edu/\-Modeling/\-G\-G\-Zmat.\-html} or {\tt http\-://www.\-arl.\-hpc.\-mil/\-P\-E\-T/cta/ccm/training/tech-\/notes/model/node4.\-html} for a description of what such a matrix is. \-The advantage of this description is that it will now be possible to describe molecules using relevant parameters (separating bond lengths, bond angles and dihedral angles), and that polyhedra (for inorganic crystals) can also be described by it. \item \-Special positions-\/\-Dynamical \-Correction \-Added a function to compute a 'smooth' correction of population as several atoms overlap (this is a dynamical correction during model search) \item \-Compilation \& \-Makefiles \-Cleaned up directories \& makefile to minimize platform dependencies. \-All platform-\/dependant commands should be in rules.\-mak, in the top directory.\end{DoxyItemize}

\end{DoxyParagraph}
\begin{DoxyParagraph}{0.3 (2000, november 24) \-: }
\begin{DoxyItemize}
\item \-Refinable\-Par \-Now \-Scatterer, \-Diffraction\-Data and \-Crystal inherit are children of \-Refinable\-Par\-List, which makes the 'refining' methods easier to write \& use. \item \-Compilation on \-Win32-\/\-Cygwin \-Compilation works using the win32 port of gcc by cygnus \-: {\tt http\-://www.\-cygwin.\-com}.\end{DoxyItemize}

\end{DoxyParagraph}
\begin{DoxyParagraph}{0.2 (2000, november 07) \-: }
\begin{DoxyItemize}
\item \-Space\-Group \-Moved to \-Sg\-Lite package, thanks to \-R. \-Grosse-\/\-Kunstleve. \item \-Powder \-Diffraction \-Most of features needed for 'basic' powder profile spectrum generation/analysis has been added (\-Diffraction\-Data\-Powder). \item \-Least-\/\-Squares \-Added an \-L\-S\-Q\-Obj\-Num class, to perform \-Least-\/\-Squares \-Refinement. \-Num stands for 'numerical', since only numerical derivatives are used for that object. \-Uses eigenvalue filtering to avoid correlations. \item \-Diffraction\-Data is now an abstract base class, with children \-Diffraction\-Data\-Single\-Crystal, \-Diffraction\-Data\-Powder.\end{DoxyItemize}

\end{DoxyParagraph}
\begin{DoxyParagraph}{0.1 \-:}
\begin{DoxyItemize}
\item \-General \-The basic classes (\-Space\-Group, \-Crystal, \-Scatterer, \-Scatterer\-:\-Atom, \-Scatterer\-:\-Regular\-Polyhedra, \-Diffraction\-Data) are there. \item \-Speed \-Computing of structure factors works with a fair speed. \end{DoxyItemize}

\end{DoxyParagraph}
